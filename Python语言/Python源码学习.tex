\documentclass[12pt]{article}
\usepackage{geometry}                % See geometry.pdf to learn the layout options. There are lots.
\geometry{letterpaper}                   % ... or a4paper or a5paper or ... 
%\geometry{landscape}                % Activate for for rotated page geometry
%\usepackage[parfill]{parskip}    % Activate to begin paragraphs with an empty line rather than an indent
\usepackage{graphicx}
\usepackage{amssymb}
\usepackage{amsmath}

%%%%%%%%%%%%%%%%%%%%%%

\usepackage{newtxtext}
\usepackage{geometry}
\usepackage{lipsum} % 该宏包是通过 \lipsum 命令生成一段本文,正式使用时不需要引用该宏包
\usepackage[dvipsnames,svgnames]{xcolor}
\usepackage[strict]{changepage} % 提供一个 adjustwidth 环境
\usepackage{framed} % 实现方框效果

\geometry{a4paper,centering,scale=0.8}
% environment derived from framed.sty: see leftbar environment definition
\definecolor{formalshade}{rgb}{0.95,0.95,1} % 文本框颜色
\definecolor{greenshade}{rgb}{0.90,0.99,0.91} % 绿色文本框,竖线颜色设为 Green
\definecolor{redshade}{rgb}{1.00,0.90,0.90}% 红色文本框,竖线颜色设为 LightCoral
\definecolor{brownshade}{rgb}{0.99,0.97,0.93} % 莫兰迪棕色,竖线颜色设为 BurlyWood

% ------------------******-------------------
% 注意行末需要把空格注释掉,不然画出来的方框会有空白竖线
\newenvironment{formal}{%
\def\FrameCommand{%
\hspace{1pt}%
{\color{DarkBlue}\vrule width 2pt}%
{\color{formalshade}\vrule width 4pt}%
\colorbox{formalshade}%
}%
\MakeFramed{\advance\hsize-\width\FrameRestore}%
\noindent\hspace{-4.55pt}% disable indenting first paragraph
\begin{adjustwidth}{}{7pt}%
\vspace{2pt}\vspace{2pt}%
}
{%
\vspace{2pt}\end{adjustwidth}\endMakeFramed%
}
% ------------------******-------------------


\usepackage[ruled]{algorithm2e} % 算法包
\usepackage{listings}
\usepackage{framed} % 解决代码分页之后代码框不闭合的问题
\usepackage{booktabs} % 三线表

%%%%%%%%%%%%%%%%%%%%%%%
%% listings设置
%%%%%%%%%%%%%%%%%%%%%%%
\lstset{
    basicstyle = \small\ttfamily,           % 基本样式 + 小号字体
    breaklines = true,                  % 代码过长则换行
    frame = shadowbox,                  % 用(带影子效果)方框框住代码块
    showspaces = false,                 % 不显示空格
    columns = fixed,                    % 字间距固定
}

\usepackage{ctex}
% 设置中文字体
\setCJKmainfont {Kai Regular}
% 设置正文罗马族的CJK字体,影响\rmfamily和\textrm 的字体
\setCJKsansfont {Kai Regular}
% 设置正文无衬线族的CJK字体,影响\sffamily和\textsf 的字体
\setCJKmonofont {Kai Regular}
% 设置正文等宽族的CJK字体,影响\ttfamily 和 \texttt 的字体


\usepackage{fontspec,xltxtra,xunicode}
\defaultfontfeatures{Mapping=tex-text}
\setromanfont[Mapping=tex-text]{Hoefler Text}
\setsansfont[Scale=MatchLowercase,Mapping=tex-text]{Gill Sans}
\setmonofont[Scale=MatchLowercase]{Andale Mono}


\usepackage{fontspec,xltxtra,xunicode}
\defaultfontfeatures{Mapping=tex-text}
\setromanfont[Mapping=tex-text]{Hoefler Text}
\setsansfont[Scale=MatchLowercase,Mapping=tex-text]{Gill Sans}
\setmonofont[Scale=MatchLowercase]{Andale Mono}



\title{Python源码学习}
\author{KenshinLiu}
\date{Last Update: \today}                                           % Activate to display a given date or no date

\begin{document}
\maketitle
\newpage

\section*{编译Python}

\begin{lstlisting}[caption={编译使用的操作系统和 Python 版本信息}]
$ sw_vers
ProductName:	macOS
ProductVersion:	12.3
BuildVersion:	21E230

$ git branch
* main

$ git log -n 1
commit df9f7597559b6256924fcd3a1c3dc24cd5c5edaf (HEAD -> main, origin/main, origin/HEAD)
Author: Inada Naoki <songofacandy@gmail.com>
Date:   Tue Mar 1 10:27:20 2022 +0900
    compiler: Merge except_table and cnotab (GH-31614)
    
\end{lstlisting}


\begin{lstlisting}[caption={编译命令}]
# debug 模式编译
$ ./configure --with-pydebug
$ make
\end{lstlisting}


\begin{lstlisting}[caption={doxygen生成调用关系}]
$ doxygen -g doxygen.config
$ doxygen doxygen.config
\end{lstlisting}

\section*{Object}

Python 所有对象的基础都是 Object. Object 在堆上进行分配。每个 Object 上都有一个表示引用计数的变量,
当引用计数变为 0 之后,Object 会被回收。 Object 上有一个表示类型的变量,每个 Object 的类型在 Object
 被创建出来的时候就被固定,之后不会在改变。这个类型也是使用一个对象来进行表示,对象内部包含一个
 指针,指针指向被称为类型对象。这个类型对象包含一个表示类型的指针,指向自身。
 
 Object 一旦分配之后,分配的内存大小以及内存地址都不会再发生改变。如果 Object 需要保存可变大小的数据,
 那么 Object 可以持有指向可变大小数据的一个指针。

Python 中的对象总是通过 PyObject* 类型的指针进行访问。 PyObject 结构体只包含引用计数和一个类型指针。
Python 中的对象的内存地址的前部都是一个 PyObject 结构体,通过这个 PyObject 结构体,就可以知道一个对象的引用计数和真实的类型。

\begin{lstlisting}[caption={PyObject定义}]
struct _object {
    _PyObject_HEAD_EXTRA // 这个字段用来编译一个 debug 版本,不用关注
    Py_ssize_t ob_refcnt;
    PyTypeObject *ob_type;
};
typedef struct _object PyObject;
\end{lstlisting}

PyObject 只能表示单个的对象,如果需要表示含有多个对象的结构,需要用到另一个结构体 PyVarObject。其中的ob\_size 字段表示对象的数量。

\begin{lstlisting}[caption={PyVarObject 定义}]
typedef struct {
    PyObject ob_base;
    Py_ssize_t ob_size; /* Number of items in variable part */
} PyVarObject;
\end{lstlisting}


上面提到了 Object 上有一个表示类型的变量。这个变量就是 PyObject 中的 ob\_type 字段,类型为 PyTypeObject。Object 真正的类型就存储在这个结构体中。

\begin{lstlisting}[caption={PyTypeObject 的定义}]
struct _typeobject {
    PyObject_VAR_HEAD
    const char *tp_name; /* For printing, in format "<module>.<name>" */
    // ... ...
};

typedef struct _typeobject PyTypeObject;
\end{lstlisting}

定义中可以看到有一个 tp\_name 字段,这个字段以字符串形式存储了类型名。

在定义的开始位置还可以看到一个 PyObject\_VAR\_HEAD,也就是上面提到的 PyVarObject,这说明 PyTypeObject 对象自身也有一个指针指向一个 PyTypeObject 类型的对象。那么这里就会有一个疑问:其他的 Object 的类型名存储在 PyTypeObject 的实例中,那么 PyTypeObject 实例自身的类型又是什么?存储在什么位置?要解答这个问题需要看一下 PyTypeObject 的实例化过程,下面以 Python 中最基础的类型类 class type 来举例子。 


\begin{lstlisting}[caption={PyTypeObject 实例化}]
PyTypeObject PyType_Type = {
    PyVarObject_HEAD_INIT(&PyType_Type, 0) // 对象的指针指向自身
    "type",                                     /* tp_name */
    // ... ...
};
\end{lstlisting}

PyType\_Type 就是 class type 的实例,在实例化的代码中可以看到 PyType\_Type 对象的 PyObject\_VAR\_HEAD 字段设置为了它自身,tp\_name 字段设置为 "type"。这样子就成功的让类的继承体系中的每个类型都有了一个类型定义。

使用如下代码可以进行验证。

\begin{lstlisting}[caption={修改 typeobject.c}]
PyTypeObject PyType_Type = {
    PyVarObject_HEAD_INIT(&PyType_Type, 0) // 对象的指针指向自身
    "PyTypeObject",   // tp_name 从 "type" 修改为 "PyTypeObject"
    // ... ...
};
\end{lstlisting}

\begin{lstlisting}[caption={创建 test.py}]
class A(object):
    pass

class B(A):
    pass

a = A()
b = B()

print(A.__class__)
print(B.__class__)

print(a.__class__)
print(b.__class__)
\end{lstlisting}

重新编译修改后的 Python,并使用生成的 Python 解释器执行 test.py,可以看到如下输出。

\newpage
\begin{lstlisting}[caption={test.py 运行输出}]
# 修改 Python 源码前, 命令 python test.py 的输出
<class 'type'>
<class 'type'>
<class '__main__.A'>
<class '__main__.B'>

# 修改 Python 源码后,命令 python test.py 的输出
<class 'PyTypeObject'>
<class 'PyTypeObject'>
<class '__main__.A'>
<class '__main__.B'>
\end{lstlisting}

从输出中可以看出,类型 A 和类型 B 的类型都变成了修改后的 “PyTypeObject”,但是对象 a 和 对象 b 的类型不变。

再看一下 Python 中 int 类型的实例化。

\begin{lstlisting}[caption={int 类型类的实例化}]
PyTypeObject PyLong_Type = {
    PyVarObject_HEAD_INIT(&PyType_Type, 0) // class int 的类型也是 class type
    "int",                                      /* tp_name */
    offsetof(PyLongObject, ob_digit),           /* tp_basicsize */
    sizeof(digit), 
}

\end{lstlisting}




\end{document}























