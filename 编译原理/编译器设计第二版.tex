\documentclass[12pt]{article}
\usepackage{geometry}                % See geometry.pdf to learn the layout options. There are lots.
\geometry{letterpaper}                   % ... or a4paper or a5paper or ... 
%\geometry{landscape}                % Activate for for rotated page geometry
%\usepackage[parfill]{parskip}    % Activate to begin paragraphs with an empty line rather than an indent
\usepackage{graphicx}
\usepackage{amssymb}
\usepackage{ctex}
\usepackage{amsmath}

%%%%%%%%%%%%%%%%%%%%%%

\usepackage{newtxtext}
\usepackage{geometry}
\usepackage{lipsum} % 该宏包是通过 \lipsum 命令生成一段本文,正式使用时不需要引用该宏包
\usepackage[dvipsnames,svgnames]{xcolor}
\usepackage[strict]{changepage} % 提供一个 adjustwidth 环境
\usepackage{framed} % 实现方框效果

\geometry{a4paper,centering,scale=0.8}
% environment derived from framed.sty: see leftbar environment definition
\definecolor{formalshade}{rgb}{0.95,0.95,1} % 文本框颜色
% ------------------******-------------------
% 注意行末需要把空格注释掉,不然画出来的方框会有空白竖线
\newenvironment{formal}{%
\def\FrameCommand{%
\hspace{1pt}%
{\color{DarkBlue}\vrule width 2pt}%
{\color{formalshade}\vrule width 4pt}%
\colorbox{formalshade}%
}%
\MakeFramed{\advance\hsize-\width\FrameRestore}%
\noindent\hspace{-4.55pt}% disable indenting first paragraph
\begin{adjustwidth}{}{7pt}%
\vspace{2pt}\vspace{2pt}%
}
{%
\vspace{2pt}\end{adjustwidth}\endMakeFramed%
}
% ------------------******-------------------





\usepackage[ruled]{algorithm2e} % 算法包

\usepackage{listings}
\usepackage{framed} % 解决代码分页之后代码框不闭合的问题

\usepackage{booktabs} % 三线表


%%%%%%%%%%%%%%%%%%%%%%%
%% listings设置
%%%%%%%%%%%%%%%%%%%%%%%
\lstset{
    basicstyle = \small\ttfamily,           % 基本样式 + 小号字体
    breaklines = true,                  % 代码过长则换行
    frame = shadowbox,                  % 用(带影子效果)方框框住代码块
    showspaces = false,                 % 不显示空格
    columns = fixed,                    % 字间距固定
}

% Will Robertson's fontspec.sty can be used to simplify font choices.
% To experiment, open /Applications/Font Book to examine the fonts provided on Mac OS X,
% and change "Hoefler Text" to any of these choices.

\usepackage{fontspec,xltxtra,xunicode}
\defaultfontfeatures{Mapping=tex-text}
\setromanfont[Mapping=tex-text]{Hoefler Text}
\setsansfont[Scale=MatchLowercase,Mapping=tex-text]{Gill Sans}
\setmonofont[Scale=MatchLowercase]{Andale Mono}

\title{《编译器设计 第二版》笔记}
\author{KenshinLiu}
\date{Last Update: \today}                                           % Activate to display a given date or no date

\begin{document}
\maketitle

\newpage

\section*{编译器检查变量在使用前已经声明}

要达到的效果:要求某些类别的变量在使用前已经声明,但允许程序员将声明和可执行语句混合起来。

解决方案:编译器创建一个名字表。编译器在处理声明时向表中插入一个名字,而在每次引用名字时去表中查找,查找失败就表示缺少对应的声明。


\section*{类型系统判断类型是否等价的两种方案}

\textbf{名字等价:} 该规则断言两个类型等价的充分必要条件是二者同名,认为相同的类型名即代表同一种类型。

\textbf{结构等价性:} 该规则断言两个类型等价的充分必要条件是二者具有相同的结构。如果两个对象由同一组字段组成,且字段排列顺序相同,对应的字段具有等价的类型,则这两个对象为同一种类型。

\section*{属性求值的方法}

(1) \textbf{动态方法} 这种技术使用特定的属性化语法分析树的结构,来确定求值次序。Knuth关于属性语法的原始论文提出了一种求值程序,以类似于数据流计算机体系结构的方式运作,即每个规则在其所有操作数就绪后即“击发”。实际上,这可以使用就绪属性(即可求值的属性)的队列来实现。随着对每个属性的求值,求值程序会检查其在属性依赖关系图中的后继属性,判断后继属性 “就绪” 与否(参见12.3节)一种相关的方案是建立属性依赖关系图,对其拓扑排序,使用拓扑次序对属性进行求值。

(2)\textbf{无关方法} 在这一类方法中,求值的次序与属性语法和特定的属性化语法分析树都是无关的。大体上,系统的设计者可以从其自身的考虑出发,选择一种他认为适合于属性语法和求值环境的方法。这种风格的求值方法包括:从左到右重复多趟(直至所有属性的值都确定为止)、从右到左重复多趟和从左到右与从右到左交替多趟处理。这些方法有简单的实现,其运行时开销也相对较小。当然,它们也缺之根据对特定属性语法树的认识进行改进的能力。

(3)\textbf{基于规则的方法} 基于规则的方法依赖于对属性语法的静态分析,来构造出一个求值次序。在该框架下,求值程序依赖于语法结构,因而,对规则的应用受到了语法分析树的引导。在有符号进制数的示例中,对产生式4的求值次序应该使用第一个规则设置Bit.position,递归向下到Bit, 返回后,使用Bit value设置List.value。。类似地,对于产生式5,它应该首先对前两个规则求值,以便为产生式的右侧定 义position屆性,然后递归向下来处理各个子结点。在返回后,就可以对第三个规则求值,来设置父结点List的List.value字段。如果工具能够离线执行必要的静态分析,那么可以利用这种工具来生成快速的基于规则的求值程序。

\section*{DAG}

DAG 是指有向非循环图。在 DAG 中,结点可以有多个父结点,相同子树可以被重用,这种共享使得 DAG 更为紧凑。

DAG 是具有共享机制的一种 AST。

\section*{控制流图}

程序中最简单的控制流单位是一个基本程序块,即(最大长度的)无分支代码序列。它开始于一个有标号的操作,结束于一个分支、跳转或条件判断操作。

控制流图(Control-Flow Graph, CFG) 用一个结点表示每个基本程序块,用一条边表示块之间的每个可能 的控制转移。

\section*{词法作用域和动态作用域}

词法作用域和动态作用域之间的区别,只出现在过程引用在自身作用域之外声明的变量时,这种变量通常称为\textbf{自由变量}。

\textbf{词法作用域规则:} 自由变量绑定到词法上与使用位置最接近的同名声明。如果编译器从包含使用处的作用域开始处理,并连续检查外层的作用域,变量将绑定到找到的第一个声明。声明总是来自包含引用处的一个作用域中。

\textbf{动态作用域规则:} 自由变量绑定到在运行时最近创建的同名变量。因此,在执行遇到自由变量时,即将绑定到改名字最新创建的实例。

\end{document}






















